% Non-breakable hyphen. Use for things like "(re-)print". 
\newcommand\nobrkhyph{\mbox{-}}

% Small caps serif font, for Paper numbers, ISBN, etc
\newcommand{\I}{\textrm{\scshape i}\xspace}
\newcommand{\II}{\textrm{\scshape ii}\xspace}
\newcommand{\III}{\textrm{\scshape iii}\xspace}
\newcommand{\IV}{\textrm{\scshape iv}\xspace}
\newcommand{\V}{\textrm{\scshape v}\xspace}
\newcommand{\VI}{\textrm{\scshape vi}\xspace}
\newcommand{\VII}{\textrm{\scshape vii}\xspace}
\newcommand{\VIII}{\textrm{\scshape viii}\xspace}
\newcommand{\IX}{\textrm{\scshape ix}\xspace}
\newcommand{\X}{\textrm{\scshape x}\xspace}
\newcommand{\ISBN}{\textrm{\scshape isbn}\xspace}
\newcommand{\ISSN}{\textrm{\scshape issn}\xspace}

% Must come in the beginning. Changes the spacing in the table of contents to look more pleasing
\usepackage{tocloft}
\setlength{\cftbeforepartskip}{5.0mm}
\setlength{\cftbeforechapskip}{2.0mm}
\setlength{\cftbeforesecskip}{0.0mm}

% Must come in the beginning. To calculate the total number of pages
\usepackage{pageslts}

%%%%%%%%%%%%%%%%%%%%%%%%%% Everything font-related %%%%%%%%%%%%%%%%%%%%%%%%%%%%%%%%%%%%%%%
\usepackage{ifxetex}

% Font selections - they are only used when you compile with xelatex (strongly recommended). If you use 
% pdflatex, the default latex fonts will be used, you will break the University graphical profile, and your
% thesis will look much much uglier.
%\usepackage[adobe-garamond]{mathdesign}
\ifxetex
	\usepackage{mathspec}
	\usepackage[no-math]{fontspec}
	\usepackage{xunicode}
	\usepackage{xltxtra}
    %\usepackage[adobe-garamond]{mathdesign}
    \usepackage[Adobe Garamond]{mathdesign}
	% Adobe Garamond Pro for serif fonts and maths, Frutiger as sans, according to LU graphical profile
	% Tables get monospaced numerals, text gets proportional numerals, mathmode gets uppercase monospaced numberals
	%\setmainfont[Mapping=tex-text,Numbers={OldStyle,Proportional}]{Adobe Garamond Pro}
	\setmainfont{AGaramondPro}[Path = ./fonts/AGaramondPro/, Mapping = tex-text, Numbers = {OldStyle, Proportional}, UprightFont = *-Regular.otf, BoldFont= *-Semibold.otf, ItalicFont=*-Italic.otf]
	\AtBeginEnvironment{tabular}{\addfontfeatures{Numbers={Monospaced}}}
	%\setsansfont{Frutiger LT Std 45 Light}
	%\setsansfont{FrutigerLTStd}[Path = ./fonts/Frutiger/, RomanFont=*-Roman.otf, BoldFont=*-Bold.otf, ItalicFont=*-Italic.otf, LightFont=*-Light.otf, LightItalicFont=*-LightItalic.otf]
	\setsansfont{FrutigerLTStd-Roman.ttf}[Path = ./fonts/Frutiger/, BoldFont=FrutigerLTStd-Bold.ttf, ItalicFont=FrutigerLTStd-Italic.ttf]
	
	%\setmathfont(Greek,Digits,Latin){Adobe Garamond Pro}
	%\setmathfont(Greek,Digits,Latin){AGaramondPro-Roman.otf}
	% mathspec is broken. The next eight lines work around that.
	\usepackage{etoolbox}
	\makeatletter
	\begingroup\lccode`~=`"
	\lowercase{\endgroup
	  \everymath{\let~\eu@active@quote}
	  \everydisplay{\let~\eu@active@quote}
	}
	\makeatother
\else
	% This is what is used for pdf latex. 
	% You can try \usepackage{ebgaramond} with (pdf)latex, however that does not have bold fonts ... !
	%\usepackage{ebgaramond}
	 \usepackage[utf8]{inputenc}
\fi
%%%%%%%%%%%% end font selections

% Support for greek (non-math non-italic) text, e.g. for units like mircometer: \textmu m 
\usepackage{textgreek}

% Additional font sizes \HUGE and \ssmall, needed for figure captions
\usepackage{moresize}

% figure captions in bold (i.e. "Figure 1" in bold), sans serif, smaller font size, hanging label, always starting on the left side
\usepackage{subfig}
\DeclareCaptionFont{ssmall}{\ssmall}
\DeclareCaptionFont{tiny}{\tiny}% "scriptsize" is defined by floatrow, "tiny" not
\captionsetup{margin=0em,font={ssmall,sf},labelfont={bf},format=hang,singlelinecheck=false} 

% figures centred, smaller font in tables, captions on top for tables
\usepackage{floatrow}
\DeclareFloatFont{tiny}{\tiny}% "scriptsize" is defined by floatrow, "tiny" not
\DeclareFloatFont{ssmall}{\ssmall}
\floatsetup[table]{font={footnotesize},position=top}
%%%%%%%%%%%%%%%%%%%%%%%%%%%%%%%%%%%% end fonts %%%%%%%%%%%%%%%%%%%%%%%%%%%%%%%%%%%%%%%%%%%%%%%%%%%%%%%

% For tables spanning the full text width
\usepackage{tabularx}

% For URLs use \url{<URL>}
\usepackage{url}

% Chapters should have numbers - a typical thesis consists of two chapters:
% One to introduce and summarize the research ("kappa"), and one for
% reproductions of the papers and manuscripts. No need to number them by default.
% If you need chapter numbers back, comment the following line.
%\renewcommand{\thesection}{\arabic{section}} 

% Sections and subsections have numbers, subsubsections etc do not.
\setcounter{secnumdepth}{3}

% Only chapters and sections appear in the table of contents, not subsections etc
\addtocontents{toc}{\protect\setcounter{tocdepth}{3}}

% Ensures that \cleardoublepage inserts empty pages without page number
\usepackage{emptypage} 

% For text macros (e.g., small caps macros)
\usepackage{xspace}

% For "Lorem Ipsum" style place holders
\usepackage{blindtext}

% For \includegraphics
\usepackage{graphicx}

% Nice looking tables with correct spacings 
\usepackage{booktabs}

% Tables spanning more than one page
\usepackage{longtable}
 
% Math symbols
\usepackage{amsfonts,mathrsfs,amsmath,amssymb}
\usepackage[ruled,vlined]{algorithm2e}
% To include the PDFs of the papers
\usepackage{pdfpages}

% For Natural Science students who use ADS. Save to have in any case.
\usepackage{auxiliary_textfiles/aas_macros}

% Hyphenation, support for different languages, last one is default
% Make sure to install the hyphenation packages for all languages you need
\usepackage[swedish,ngerman,english]{babel} 

% Reference in style of Natural Sciences
\usepackage[super, comma, sort]{natbib}

% Non-indented paragraphs with a small vertical space in-between
\parindent 0in
\parskip 3mm

% Paper size. Typically this works also fine when printed on A4 (text size is
% as print later, just the margins become wider to accommodate the too large
% sheets).
%% G5 format
%\usepackage[paperwidth=169mm,paperheight=239mm,total={13.3cm,19.6cm}, top=1.8cm, ignorehead, centering, footskip=\footskip+4mm ]{geometry}
\usepackage[paperwidth=169mm,paperheight=239mm,total={12.5cm,19.6cm}, top=1.8cm, ignorehead, centering, footskip=\footskip+4mm ]{geometry}
% If you prefer E5 (smaller) you can use this line instead. Also go to frontmatter.tex and change the datasheet from G5 to E5.
%% E5 format
%\usepackage[paperwidth=155mm,paperheight=220mm,total={12cm,18cm}, top=1.8cm, ignorehead, centering, footskip=\footskip+4mm ]{geometry}

% not every page needs to go to the same bottom line. Allows nicer page breaks.
\raggedbottom

% avoid orphan/widow lines. Lower this number if necessary to get a good layout.
\widowpenalty500
\clubpenalty500


% These command should allow figures to be placed close to where we define them, 
% thus we can influence the figure placement to some extent.
%% min page fraction that must be filled with text
%\renewcommand{\textfraction}{0.1} 
\renewcommand{\textfraction}{0.2}
%% max page fraction that a float may take at the top of the page
%\renewcommand{\topfraction}{0.9} 
\renewcommand{\topfraction}{1.0}
%% max page fraction that a float may take at the bottom of the page
%\renewcommand{\bottomfraction}{0.9}
\renewcommand{\bottomfraction}{1.0}
%% max page fraction that may be filled with floats
%\renewcommand{\floatpagefraction}{0.5}
\renewcommand{\floatpagefraction}{1.0}
%% maximum number of floats at the top of the page
\setcounter{topnumber}{1}
%% maximum number of floats at the bottom of the page
\setcounter{bottomnumber}{1}
%% maximum total number of floats on a page
\setcounter{totalnumber}{1}

% Figures can be kept in the same directory as the thesis, or in a Figure/
% directory, without need to specifying which of the two it is
\graphicspath{{figures/}}


%%%%%%%%%%%%%%%%%%%%%%%%%%%%%%%%%%%%%%%%%%%%%%%%%%%%%%%%%%%%%%%%%%%%
% define commands for chapters, sections, and subsections that do not 
% have a number, but are entered as candidates for the table of contents.
\newcommand\chap[1]{%
  \chapter*{#1}%
  \addcontentsline{toc}{chapter}{#1}
  \markboth{#1}{#1}
}
\newcommand\sect[1]{%
  \section*{#1}%
  \addcontentsline{toc}{section}{#1}
  \markright{#1}
}
\newcommand\subsect[1]{%
  \subsection*{#1}%
  \addcontentsline{toc}{subsection}{#1}
}
\newcommand\subsubsect[1]{%
  \subsubsection*{#1}%
  \addcontentsline{toc}{subsubsection}{#1}
}


%%%%%%%%%%%%%%%%%%%%%%%%%%%%%%%%%%%%%%%%%%%%%%%%%%%%%%%%%%%%%%%%%%%%
% Define nice headers and footers
% To keep the thesis non-cluttered we only put the page number into the footer,
% and avoid headers
\usepackage{fancyhdr}
\fancyheadoffset{0cm}
\pagestyle{plain}
%page number in the foot centre 
\cfoot{\fancyplain{\thepage}{}}

% You want to use headers? Here is some things for your inspiration:
%\renewcommand{\chaptermark}[1]%
%               {\markboth{#1}{#1}}
%\renewcommand{\sectionmark}[1]%
%               {\markright{\thesection\ #1}}
%\renewcommand{\chaptermark}[1]{\markboth{\thechapter\ #1}{\thechapter\ #1}}
%\lhead[\fancyplain{}{\thepage}]%
%      {\fancyplain{}{\textsl\rightmark}}
%\rhead[\fancyplain{}{\textsl\leftmark}]%
%      {\fancyplain{}{\thepage}}
% Contents and references should not print a uppercase header on their second pages
%\usepackage{etoolbox}
%\patchcmd{\tableofcontents}{\MakeUppercase\contentsname}{\contentsname}{}{}
%\patchcmd{\tableofcontents}{\MakeUppercase\contentsname}{\contentsname}{}{}
%\patchcmd{\bibsection}{\MakeUppercase{\bibname}}{\bibname}{}{}
%\patchcmd{\bibsection}{\MakeUppercase{\bibname}}{\bibname}{}{}

% References should be a section of the summary text, with number and all ...
\renewcommand\bibsection{\chap{\bibname}\markright{\bibname}}

% If you want to use a box to summarize your papers, try \begin{thesisbox}{title}
\usepackage{tcolorbox}
\tcbuselibrary{skins}
\newtcolorbox{thesisbox}[2][]{
oversize=-2.5cm,
  enhanced,
  attach boxed title to top left={yshift=-2ex,xshift=6ex},
  before skip=1em,
  after skip=1em,
  colframe=black,
  colback=white,
  fonttitle=\bfseries, 
  colbacktitle=white,
  coltitle=black,
  boxed title style={
    boxrule=0pt,
    colframe=white,
    },
  title=#2,
  #1}
\newcounter{exa}