\documentclass{article}
\begin{document}

\section{Own Contributions}

\textbf{Oliver Matonoha}\\

These are the publications with the author's contribution, in chronological order:

\begin{itemize}
\item J.~Adolfsson \textit{et al.} The upgrade of the ALICE TPC with GEMs and continuous readout.
JINST \textbf{16} (2021) no.03, P03022,
[arXiv:2012.09518 [physics.ins-det]]. \\\textit{Summary paper of the TPC upgrade by the ALICE TPC collaboration, of which I am a member, as part of my qualification task related to disassembling of the front-end readout system and the upgraded TPC commissioning.}

\item O.~Matonoha. Light-flavour hadron production as a function of the underlying event. \textit{ARISF} 2021, p.\ 277. Available at \\\textit{https://cds.cern.ch/record/2758268}. \\\textit{Proceedings of the 55th Recontres de Moriond on QCD conference.}

\item J.~Adolfsson, \textit{et al.}
QCD challenges from pp to A\textendash{}A collisions,
Eur. Phys. J. A \textbf{56} (2020) no.11, 288
[arXiv:2003.10997 [hep-ph]] \\\textit{Summary paper to the 3rd International Workshop on QCD Challenges from pp to A--A, 2019, presenting main ideas discussed during the workshop, where I actively participated and helped review the final document.}

\item ALICE Collaboration. Production of pions, kaons and protons as a function of the transverse event activity in pp collisions at $\sqrt{s}=13$ TeV. Accepted by JHEP, preprint [arXiv:2301.10120 [nucl-ex]]. \\\textit{The first publication of my collaborators and I on the underlying event measurements. I was one of four members on the Paper Committee, which together developed the analysis techniques and the use of the observable $R_\mathrm{T}$. The publication presents particle spectra of pions, kaons, and protons, primarily analysed by my colleague O.\ Vazquez. }

\item ALICE Collaboration. Light-flavor particle production in high-multiplicity pp collisions at $\sqrt{s}=13$ as a function of transverse spherocity. Publication currently undergoing the internal review process of ALICE. \\\textit{Publication of my collaborators and I on the spherocity measurements, including results presented in this dissertation. I was one of six members on the Paper Committee, which together developed the analysis techniques and the use of the observable $S_O^{p_\mathrm{T}=1}$. My primary responsibility was the analysis of the neutral kaon and Lambda particles.}

\end{itemize}

ALICE publications have more than one thousand authors and are subjected to an extensive internal review process, often taking several years.
The Paper Committee in ALICE is the group of collaborators responsible for conducting the actual data analysis, producing the figures, and writing of the final publication.
The neutral strange hadron measurements as a function of underlying event activity presented in this thesis have not been published yet but are planned for publication in the near future.

\end{document}