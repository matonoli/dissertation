\section{Summary of the research goals}

The first part of this dissertation provides an introduction to quantum chromodynamics (QCD), a theory of the fundamental strong force of the Universe, and explains why QCD interactions involving low-momentum (soft) transfers cannot be calculated from first principles, unlike hard processes, which can use perturbation theory. Next, the importance of studying QCD matter at extreme conditions is discussed, specifically the plasma of deconfined quarks and gluons QGP. 

Furthermore, the dissertation presents an overview of experimental studies that investigate the properties and signatures of the QGP in AA collisions and their dependence on collision centrality, which can be related to energy densities as well as final-state multiplicities in the system. Special attention is given to the phenomena of strangeness enhancement, where strange particles are produced more abundantly in more active collisions, and collective flow, where the hydrodynamic behavior of the plasma affects the dynamics of final-state hadrons.

The dissertation also describes the challenges to traditional paradigms that pp collisions were thought to be incapable of producing extremes of QCD matter, and lists the observations resembling traditional QGP phenomena in these small system collisions. Moreover, it explains the intricacies of isolating the physics behind these phenomena, given that event multiplicity arises from non-perturbative (softer) as well as perturbative (harder) processes, is susceptible to large fluctuations, and cannot be directly linked to the energy density.

The goal of this dissertation is to provide a clearer and more differential study of QGP phenomena, namely strangeness enhancement and radial flow, and to elucidate the roles played by both soft and hard processes. The aim is to identify and utilise observables that isolate events with extreme activity resulting from non-perturbative processes, where novel physics may be at play, such as the formation of a QGP-like state or complex interactions of overlapping QCD fields. Two state-of-the-art phenomenological models are used to represent these paradigms: EPOS LHC, which includes QGP droplets, and the Ropes tune of Pythia 8, which allows for the merging of strings into higher-tension fields. These observables should also be capable of isolating event activity extremes dominated by perturbative physics at the other end of the spectrum.

Given the complexity of the physics picture and its need to be studied from various angles, these measurements are unlikely to single-handedly confirm or reject the ``big hypothesis" of whether QGP can be formed in pp collisions. Nevertheless, they will be able to provide valuable insight into the underlying physics processes in hadronic and partonic interactions and their deeper understanding. Furthermore, the measurements have the potential to significantly discriminate between different phenomenological models.

\section{Highlights of the \SOPT measurement}

Chapter~\ref{chap:sphero} introduces measurements of the neutral, weakly decaying \KOs, \LA, and \AL as a function of transverse spherocity \SOPT in pp collisions at \sppt{13} using the ALICE detector at the LHC. This observable describes the geometrical shape of the charged particles produced in the collision and provides a simple, albeit effective discriminator between pencil-like events, dominated by a di-jet coming from a single hard partonic scattering, and isotropic events, where particles are produced from multiple sources involving lower momentum transfers.

These measurements provide the first ever experimental results of these strange particles as a function of an event shape observable. Moreover, they develop and utilise the so-called unweighted spherocity, a modification from its traditional form \SO, which is easier to compare with phenomenological models. The results are fully corrected and come with a thorough inverstigation of their experimental uncertainties.

In this dissertation, the following results are presented: transverse momentum \pt spectra, average \meanpt, ratios of \pt spectra to pions, baryon-to-meson ratios, and integrated production yields as a function of \SOPT in high-multiplicity events. The following findings can be highlighted:
\begin{enumerate}
\item Figure~\ref{fig:sphero:v0mvscl1} shows that \SOPT has interplays with the event multiplicity:
\begin{itemize}
\item If high multiplicity is determined at forward-rapidity (\VOM), \SOPT varies mostly the mid-rapidity multiplicity, leading to more constant increases and decreases in particle spectra between jetty and isotropic events (Fig.~\ref{fig:sphero:lpt}). 
\item If high multiplicity is determined at mid-rapidity (\NSPD), \SOPT varies mostly the \meanpt, leading to similar yields but significant hardening and softening between jetty and isotropic events (Fig.~\ref{fig:sphero:lpt} and Fig.~\ref{fig:sphero:meanpt}).
\end{itemize}
\item Generally, the isotropic spectra are closer to the \SOPT-unbiased spectra than the jetty are, suggesting that average high-multiplicity collisions are somewhat isotropic, whereas jetty events are outliers.
\item Ratios to pions in Fig.~\ref{fig:sphero:ltopi} reveal that relatively to pions, \KOs and \LA are consistently suppressed in jetty events and enhanced in isotropic events in \NSPD classes. This effect is different in the \VOM class.
\item The baryon-to-meson ratios \ltok in Fig.~\ref{fig:sphero:ltok} display an enhancement of \LA in intermediate \pt in isotropic events. This is expected with observing radial flow, however, its other characteristic features, such as shift of the peak, are not seen.
\item Ratios of integrated yields to pions in Fig.~\ref{fig:sphero:lvssOpt} show a characteristic strangeness enhancement behaviour for the \NSPD class and the effect increases with increasing the mass and strangeness content. This result is the first observation ever of strangeness enhancement that occurs with (mostly) constant multiplicity. In the \VOM class, the effect is not observed.
\item Results are compared with selected MC predictions, mostly favouring Pythia 8 Ropes and EPOS LHC over Pythia 8 Monash, although with varying degrees of success.
\end{enumerate}

\section{Highlights of the \RT measurements}

Chapter~\ref{chap:rt} presents measurements of the production of \KOs, \LA, and \AL in pp collisions at \sppt{13} and their dependence on the relative underlying event activity \RT. This quantity controls the magnitude of the underlying event, which consists of many softer particles produced through multiple partonic interactions (MPIs) and other sources, and is unrelated to the primary hard partonic scatterings and its fragmentation. It acts as a dial, selecting events resembling ee collisions ($\RT\to0$, $\nmpi\to1$) and AA collisions ($\RT\to\infty$, $\nmpi\to\infty$). Particle production is studied in three azimuthal regions based on the highest-\pt track, which serves as a proxy for the primary scattering axis: Towards, Away, and Transverse.

Moreover, to distinguish between the contributions to the underlying event from softer (MPIs) and harder (wide-angle radiation ISR/FSR) interactions, this approach is extended to divide the Transverse region further into Transverse-min and Transverse-max based on the number of particles, and define the classifiers \RTmin and \RTmax, accordingly. At the time of conducting this measurement, the \RTmin observable is expected to be the cleanest probe of the \meannmpi, as the harder contributions are captured in the \RTmax quantity.

The measurements presented in this dissertation are the first ever experimental results of \KOs, \LA, and \AL as a function of \RT, as well as the first use ever of \RTmin and \RTmax on identified particles. Considerable effort has been required to experimentally use and understand these observables, particularly in the choice of tracks used, the treatment with Bayesian unfolding, and the quantification of systematic uncertainties.

The following results are focused on: \pt spectra, \meanpt, baryon-to-meson ratios, and self-normalised yields. Key outcomes of this study are:
\begin{enumerate}
\item Figure~\ref{fig:rt:meanptK0s} implies a much steeper increase of \meanpt of \KOs with \RTmax than \RTmin.
\item The transverse momentum spectra in Fig.~\ref{fig:rt:ptK0s} and Fig.~\ref{fig:rt:ptLA} show:
\begin{itemize}
\item In the Toward region, the spectra in different \RT events converge at high \pt to the \RT-integrated values, corresponding to the dominance of jet.
\item In the Transverse region, \pt spectra continue hardening with increasing \RT, similarly to the picture in multiplicity measurements.
\item In contrast, the Transverse-min region seems to plateau.
\end{itemize}
\item The baryon-to-meson ratios in Fig.~\ref{fig:rt:LtoK} reveal typical radial flow features in high-\RT events, including the enhancement of intermediate-\pt baryons, their depletion at low-\pt, as well as shifts of the peaks. Moreover,
\begin{itemize}
\item In the Toward region, the effect is the largest. This is due to the mixing of jet- and UE-related particle production, which shows largely different \ltok ratios.
\item The effect is comparable among the Transverse, Transverse-min, and Transverse-max regions, with only small hints of being slightly bigger in the Transverse-min case. This suggests that the harder and softer components of UE affect the relative production of \LA and \KOs in a similar fashion.
\end{itemize}
\item The self-normalised yields in Fig.~\ref{fig:rt:yield} of \KOs and \LA are consistent with $\pi$ and p, respectively, in the Toward and Away regions, although more experimental precision is needed. In the Transverse region, the effect of auto-correllation is apparent for the charged particles. Moreover, the \KOs and \LA yields rise more slowly in the Transverse-min cases than the Transverse-max.
\item The experimental data favour the Pythia 8 Ropes predictions. EPOS LHC does not display the right amount of sensitivity to the azimuthal region and the Pythia 8 Monash tune underestimates the effects of radial flow. This implies that Colour Reconnection is somewhat insufficient to describe the flow-like behaviour.
\item Generally, higher values of \RT/\RTmin/\RTmax observables need to be reached in order to isolate the different behaviours of softer MPIs and harder ISR/FSR and further test the MC predictions.
\end{enumerate}

\section{Conclusions and Outlooks}

\textit{One paragraph wrap-up?}
\textit{One paragraph about possible improvements}
\textit{One paragraph about other areas for exploration}